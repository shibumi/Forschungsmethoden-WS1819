\documentclass[11pt]{article}
%
% Abstract template for the IRIS-9 meeting
% in case of trouble please contact IRIS-9@mps.mpg.de.
%
% Do not exceed one page in the compiled version.
%
%%%%%%%%%%%%%%%%%%%%%%%%%%%%%%%%%%%%%%%%%%%%%%%%%%%%%%%%%%%%%%%%%%
% Do not change or extend the following definitions
% or use any style files!
\pagestyle{empty}
\newcommand{\type}[1]{Beitrag:\par#1}
\newcommand{\session}[1]{\vspace{5mm}Vorlesung:\par#1\par}
\renewcommand{\title}[1]{\vspace{5mm}{\Large\bf#1}\par}
\newcommand{\authors}[1]{\vspace{5mm}#1\par}
\newcommand{\presenting}[1]{{\underline{#1}}}
\newcommand{\affiliations}[1]{\vspace{5mm}{\small\em#1}\par}
\renewcommand{\abstract}[1]{\vspace{5mm}\parbox{\textwidth}{#1}}
\centering
\begin{document}
\centerline{\bf Technische Universität Clausthal, Clausthal-Zellerfeld, 12.01.2019}{\vspace{5mm}
%%%%%%%%%%%%%%%%%%%%%%%%%%%%%%%%%%%%%%%%%%%%%%%%%%%%%%%%%%%%%%%%%%


\type{%%%%%%%%%%%%%%%%%%%%%%%%%%%%%%-UNCOMMENT-ONE-%%%%%%%%%%%%%%%
	Poster               \par
%	Contributed Talk     \par
%	Invited Talk         \par
}%%%%%%%%%%%%%%%%%%%%%%%%%%%%%%%%%%%%%%%%%%%%%%%%%%%%%%%%%%%%%%%%%


\session{%%%%%%%%%%%%%%%%%%%%%%%%%%%-UNCOMMENT-ONE-%%%%%%%%%%%%%%%
%
	Forschungsmethoden            \par
%	2. Chromospheric heating and dynamics                     \par
%	3. Magnetic coupling and mass flux through the atmosphere \par
%	4. Eruptions in the solar atmosphere                      \par
%	5. Opportunities and challenges                           \par
%	6. Science together with future facilities                \par
%
}%%%%%%%%%%%%%%%%%%%%%%%%%%%%%%%%%%%%%%%%%%%%%%%%%%%%%%%%%%%%%%%%%


\title{%%%%%%%%%%%%%%%%%%%%%%%%%%%%%%%%%%%%%%%%%%%%%%%%%%%%%%%%%%%
Prototyping
}%%%%%%%%%%%%%%%%%%%%%%%%%%%%%%%%%%%%%%%%%%%%%%%%%%%%%%%%%%%%%%%%%


\authors{%%%%%%%%%%%%%%%%%%%%%%%%%%%%%%%%%%%%%%%%%%%%%%%%%%%%%%%%%
Sajedeh Majdi (TU Clausthal), Sajedeh.Majdi@tu-clausthal.de \\
Amin Beikzadeh (TU Clausthal), Admin.Beikzadeh@tu-clausthal.de \\
Christian Rebischke (TU Clausthal), Christian.Rebischke@tu-clausthal.de
}%%%%%%%%%%%%%%%%%%%%%%%%%%%%%%%%%%%%%%%%%%%%%%%%%%%%%%%%%%%%%%%%%


\abstract{%%%%%%%%%%%%%%%%%%%%%%%%%%%%%%%%%%%%%%%%%%%%%%%%%%%%%%%%
%
This is the \LaTeX template for the submission of abstracts to the IRIS-9 meeting to be held from 25-29 June 2018 in G\"ottingen, Germany at the Max Planck Institute for Solar System Research.

\begin{itemize}
\item 
Please replace this text with your own abstract and edit the source file in order to specify if you prefer to give a contributed talk or to present a poster. If you have been invited for a talk please uncomment the respective entry.
\item 
Please also specify your preferred session by uncommenting the respective line in the source file.
\item
Please indicate the presenting author for the contribution using the $\backslash$presenting$\{\dots\}$ command in the author template.
\item 
Please do not use any style files or change any of the header commands in this \LaTeX file (before $\backslash$begin$\{$document$\}$).
\item
Please restrict yourself to one page in the formatted abstract.
\end{itemize}
For submission simply upload your finalised (and tested) \LaTeX file source file through the workshop abstract submission webpage.
%
}%%%%%%%%%%%%%%%%%%%%%%%%%%%%%%%%%%%%%%%%%%%%%%%%%%%%%%%%%%%%%%%%%


\end{document}
