\documentclass[11pt]{article}
\usepackage{hyphenat}
%
% Abstract template for the IRIS-9 meeting
% in case of trouble please contact IRIS-9@mps.mpg.de.
%
% Do not exceed one page in the compiled version.
%
%%%%%%%%%%%%%%%%%%%%%%%%%%%%%%%%%%%%%%%%%%%%%%%%%%%%%%%%%%%%%%%%%%
% Do not change or extend the following definitions
% or use any style files!
\pagestyle{empty}
\newcommand{\type}[1]{Beitrag:\par#1}
\newcommand{\session}[1]{\vspace{5mm}Vorlesung:\par#1\par}
\renewcommand{\title}[1]{\vspace{5mm}{\Large\bf#1}\par}
\newcommand{\authors}[1]{\vspace{5mm}#1\par}
\newcommand{\presenting}[1]{{\underline{#1}}}
\newcommand{\affiliations}[1]{\vspace{5mm}{\small\em#1}\par}
\renewcommand{\abstract}[1]{\vspace{5mm}\parbox{\textwidth}{#1}}
\centering
\begin{document}
\centerline{\bf Technische Universität Clausthal, Clausthal-Zellerfeld, 12.01.2019}{\vspace{5mm}
%%%%%%%%%%%%%%%%%%%%%%%%%%%%%%%%%%%%%%%%%%%%%%%%%%%%%%%%%%%%%%%%%%


\type{%%%%%%%%%%%%%%%%%%%%%%%%%%%%%%-UNCOMMENT-ONE-%%%%%%%%%%%%%%%
	Poster               \par
%	Contributed Talk     \par
%	Invited Talk         \par
}%%%%%%%%%%%%%%%%%%%%%%%%%%%%%%%%%%%%%%%%%%%%%%%%%%%%%%%%%%%%%%%%%


\session{%%%%%%%%%%%%%%%%%%%%%%%%%%%-UNCOMMENT-ONE-%%%%%%%%%%%%%%%
%
	Forschungsmethoden            \par
%	2. Chromospheric heating and dynamics                     \par
%	3. Magnetic coupling and mass flux through the atmosphere \par
%	4. Eruptions in the solar atmosphere                      \par
%	5. Opportunities and challenges                           \par
%	6. Science together with future facilities                \par
%
}%%%%%%%%%%%%%%%%%%%%%%%%%%%%%%%%%%%%%%%%%%%%%%%%%%%%%%%%%%%%%%%%%


\title{%%%%%%%%%%%%%%%%%%%%%%%%%%%%%%%%%%%%%%%%%%%%%%%%%%%%%%%%%%%
Prototyping
}%%%%%%%%%%%%%%%%%%%%%%%%%%%%%%%%%%%%%%%%%%%%%%%%%%%%%%%%%%%%%%%%%


\authors{%%%%%%%%%%%%%%%%%%%%%%%%%%%%%%%%%%%%%%%%%%%%%%%%%%%%%%%%%
Sajedeh Majdi (TU Clausthal), Sajedeh.Majdi@tu-clausthal.de \\
Amin Beikzadeh (TU Clausthal), Admin.Beikzadeh@tu-clausthal.de \\
Christian Rebischke (TU Clausthal), Christian.Rebischke@tu-clausthal.de
}%%%%%%%%%%%%%%%%%%%%%%%%%%%%%%%%%%%%%%%%%%%%%%%%%%%%%%%%%%%%%%%%%


\abstract{%%%%%%%%%%%%%%%%%%%%%%%%%%%%%%%%%%%%%%%%%%%%%%%%%%%%%%%%
%
Im Seminar \emph{Forschungsmethoden} an der Technischen Universität Clausthal
soll den Studierenden ein Einblick in wissenschaftliches Arbeiten
gebracht werden, so dass diese am Ende ihres Master\hyp{}Studiums selbst
in der Lage sind eine wissenschaftliche Arbeit nach internationalen
Standards zu verfassen. Dazu sollen die Studierenden Gruppen bilden und ein
gemeinsames Poster zu einer Forschungsmethode erarbeiten. In diesem
Abstract möchten wir unsere gewählte Forschungsmethode
\emph{Prototyping} kurz vorstellen und eine Richtung für unser Poster
vorgeben. Im \emph{Prototyping} wird versucht möglichst frühzeitig erste
Ergebnisse zu einer Forschungsfrage zu erhalten, in dem die
\emph{Durchführbarkeit} oder die \emph{Eignung} des Forschungsobjekts
direkt anhand eines praktischen Beispiels getestet wird, ein Prototyp.
In unserem Poster möchten wir mögliche Gründe für ein \emph{Prototyping}
erläutern, sowie dessen Vor\hyp{} und Nachteile erläutern, und einige
Beispiele zum \emph{Prototyping} nennen.
%
}%%%%%%%%%%%%%%%%%%%%%%%%%%%%%%%%%%%%%%%%%%%%%%%%%%%%%%%%%%%%%%%%%


\end{document}
